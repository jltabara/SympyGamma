\documentclass[a4paper,11pt]{article}
%%%%%%%%%%%%%%%%%%%%%%Paquetes
\usepackage[spanish]{babel}  
\usepackage{indentfirst} %%%%%%%%%%%%%%%%Crear un indent al principio
\usepackage[latin1]{inputenc}%%%%%%%%%%%%� y acentos
%\usepackage{enumerate}%%%%%%%%%%%%%%%%Mejoras del entorno enumerate
\usepackage{chancery}
\usepackage[paperwidth=12cm,paperheight=15cm, hmargin={0.5cm,0.5cm}, top=1cm, bottom=1cm]{geometry}
\usepackage{latexsym,amsmath,amssymb,amsfonts}


%----- Paquete TiKz ------------------------
\usepackage{tikz}
\usepackage{enumitem}
\newcommand*{\itembolasazules}[1]{\footnotesize\protect\tikz[baseline=-3pt]%
\protect\node[scale=.75, circle, shade, ball
color=green!40!black]{\color{white}\large\bf#1};}
%---------------------------------------------

\usepackage{setspace}
\usepackage{xcolor}


\linespread{0.9}

\pagestyle{empty}



\begin{document}








\begin{enumerate}[label=\itembolasazules{}]

\item Operaci�n  con enteros:
\[
-7\cdot 5+(6-2)^7, \qquad 8976 : 65
\]

\item Operaci�n con fracciones:
\[
\left(\frac{1}{2}+\frac{2}{3}\right) \left(\frac{1}{5}\right)^7
\]

\item Operaci�n con decimales y radicales:
\[
4: 2.6-7.89(34-5.3^4), \quad \sqrt{68}, \quad  \sqrt[5]{789}
\]

\item Funciones trigonom�tricas:
\[
\sin\left(\frac{3\pi}{4}\right), \qquad \tan\left(\frac{\pi}{3}\right), \qquad \cos(45^o),\qquad \arccos(0.5)
\]


\item Logaritmos:
\[
\log_{10}(1000), \quad \log_{10}(342),\quad \log_2(32) ,\quad \ln(e^5),\quad \cos(\ln(56))
\]


\item Factorizaci�n:
\[
\mathrm{factorizar(630)}, \qquad \mathrm{isprime}(23)
\]

\item Divisores y m�ltiplos:
\[
\mathrm{divisores}(180),\qquad \mathrm{mcd}(450, 325), \qquad \mathrm{mcm}(450,325)
\]

\item Ecuaciones y sistemas:
\[
x^2-5x+6=0, \quad  x^2+4=0,\quad
\begin{cases}
4x+6y=3\\
-3x-7y=2
\end{cases}
\]

\item Operaciones con polinomios:
\[
(x-3)(4x^2-3x+1)+(x-4)^3, \qquad (x^3-4x^2-6x+2):(2x-3)
\]


\item Factorizaci�n polinomios:
\[
x^4-7x^3+17x^2-17x+6
\]

\item Mcd y Mcm de polinomios:
\[
x^4-7x^3+17x^2-17x+6, \qquad x^3-13x+12
\]

\item Fracciones algebraicas:
\[
\frac{x^2-5x+6}{x^2-6x+8}, \qquad \frac{1}{x-4}+\frac{3x-3}{x^2-1}
\]


\item Gr�ficas de funciones:
\[
\cos(x), \qquad \exp(\sqrt{x})
\]

\item N�meros complejos:
\[
\frac{3-5i}{2+7i} +(3+i)(-3), \qquad |3+6i|
\]

\item L�mites:
\[
\lim_{x\rightarrow 0} \frac{\sin(x)}{x}, \qquad \lim_{x\rightarrow \infty}\left(1+\frac{1}{x}\right)^x
\]


\item Derivadas:
\[
\frac{\mathrm{d}}{\mathrm{dx}}(5x^6), \qquad  \frac{\mathrm{d^2}}{\mathrm{dx^2}}(\cos(\ln(x)))
\]

\item Integrales:
\[
\int \sin(2x)\, \mathrm{dx} ,\qquad \int_1^4 (x^2-3x)\mathrm{dx}
\]

\item Operaciones matriciales:
\[
\begin{pmatrix}
4 & 7\\
-2 & 2
\end{pmatrix}
\cdot
\begin{pmatrix}
1 &5\\
2 &9
\end{pmatrix}
\]

\item Determinante, inversa:
\[
\begin{pmatrix}
4 & 4 & 3\\
-2 & 6& 3\\
1&3 & 4
\end{pmatrix}
\]

\end{enumerate}

\end{document}